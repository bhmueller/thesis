\section{Comparing Shapley Effects and Morris Indices} \label{comparison}

In this section I want to connect and further discuss the results presented in \cref{shapley_rust_model}
and \cref{morris_rust_model}.

Interestingly, in the case of two inputs and when setting $N_I = 3$ as recommended
by \citet{SNS16}, the computational burden of both algorithms are parallel, when
considering $N$ and $N_O$. Then they differ by $N_V$ only, given that $N = N_O$. One drawback
of the Shapley effects as in the implementation of \citet{SNS16} is that one needs to
set more than one sample size as in the case of the Morris method.
Both, Shapley effects and Morris indices, succeed in identifying the correct input
ranking, even for a relatively small sample size.

With respect to the estimation precision, Morris indices are subject to larger uncertainty
than Shapley effects. As stated above, this estimation precision is not necessarily needed. If
the identification of uninfluential inputs is the purpose of sensitivity analysis, estimates need to be precise
only in the sense that they yield nonzero indices if an input is important and zero if they
are not. If inputs ranking is the goal, the situation is slightly different. The simulation
study presented in this paper with two inputs only may not be the right trial to investigate
this. In the case of the Rust model, I conclude that input ranking is achieved, even in the
light of volatile Morris indices.

Since the estimation of Shapley effects is more costly and acknowledging that Morris
indices yield correct input rankings for the Rust model, one can resort to using the Morris
method for input importance ranking. Further, Morris indices are informative about the
model structure. Hence, one can not only successfully rank the inputs but also learn more
about the model itself.

Since Shapley effects yield only one sensitivity index per input, they are easier to interpret than the four indices in the case of the Morris method.

My findings are in line with the ones found in \citet{GM17}. The extension of the Morris method to dependent inputs is an important contribution.