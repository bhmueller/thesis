\section{Conclusion} \label{conclusion}

% Was sind die Erkenntnisse, die Du gewonnen hast.

% Wie knüpfen die an vorherige Literatur an?

When assessing the predictive validity of the extended Morris method proposed by \citet{GM17}, my main finding is that the method successfully ranks model inputs according to their importance when independent indices are considered. This holds true even for small sample sizes. It is only a valid substitute for Shapley effects if we apply sensitivity analysis for FF since full Morris indices successfully establish that both inputs of the Rust model are important, which is in line with the interpretation based on Shapley effects. My findings stress the importance of considering both, independent and full Morris indices, to identify an input as uninfluential. A conclusion based solely on the independent Morris indices can lead to Type II error in the case of the Rust model.

%It is therefore a valid substitute for Shapley effects if the purpose of sensitivity analysis is importance ranking or FF in the case of the Rust model.  My results are generally in line with the ones found in \citet{GM17}: this thesis adds one more piece of evidence in favour of the extended Morris method. This evidence hopefully encourages and facilitates the widespread application of sensitivity analysis in Economics. With the Morris method we have an easy to grasp and easy to implement sensitivity method, while performing well at low computational cost.

If FF is the purpose, none of the inputs of the Rust model can be fixed. As my results show, both Shapley effects and Morris indices indicate this. The inherent uncertainty in the model inputs should not be ignored if one is to use the Rust model for policy recommendations. Assuming that we knew either of the two inputs with certainty would lead to wrong predictions since the output is sensitive to both. If uncertainty in implied annual demand is to be reduced, my investigation suggests that particular attention should be allocated to the slope parameter of the cost function $\theta_{11}$. Independent and full Morris indices show that the influence of the replacement costs, $RC$, on implied annual demand for bus engines is due to its dependence on $\theta_{11}$ only.

% Implications.

Since the Morris method is a computationally tractable sensitivity method, it can encourage the wide-spread application of sensitivity analysis in economic modelling which then enables researchers and practitioners to build better models for policy recommendations. My results add one more piece of evidence in favour of the application of the Morris method for sensitivity analysis.

% Lassen sich die Erkenntnisse verallgemeinern oder was sind Spezifika, die zu beachten sind?

Since \citet{GM17} do not provide formal arguments about the properties of Morris indices, it is not clear whether these findings generalise. Future research on the convergence properties of the extended Morris method, especially for full indices, is needed.

% Was sind offene Fragen für zukünftige Forschung.