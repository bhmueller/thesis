\section{Qualitative Sensitivity Measures: the Morris Method}

\subsection{Input Screening}

\subsection{Morris Method for Independent Inputs}

The Morris method was introduced by \citet{M91}.

Consider the same setup as employed in the preceding sections. Let $x = \{x_1,\ \dots,\ x_k\}$ denote a sample of values assigned to the $X_i$'s. $f(x)$ is then the model output obtained for the values in $x$. Now consider a second sample $x_{\Delta_i} = \{x_1,\ \dots,\ x_{i-1},\ x_i + \Delta,\ x_{i+1},\ \dots,\ x_k\}$ that is identical to $x$ up to input $x_i$ which is varied by $\Delta$. Then, one elementary effect for input $i$ is derived by
\begin{equation}
EE_i = \frac{f(x_{\Delta_i}) - f(x)}{\Delta}.
\end{equation}

The above elementary effect is computed $N$ times, each for a varying $\Delta$ \citep{GM17}. The actual sm resulting from the Morris method are the mean, denoted by $\mu_i$, and the standard deviation, denoted by $\sigma_i$, taken from all $N$ different elementary effects per input $i$.
\begin{align}
\mu_i^\ast& = \frac{1}{N} \sum_{r=1}^N \vert EE_{i,\ r} \vert,\\
\sigma_i& = \sqrt{\frac{1}{N-1} \sum_{r=1}^N (EE_{i,\ r} - \mu_i)^2},
\end{align}

\noindent with $EE_{i,\ r}$ denoting the $r$-th elementary effect of input $i$, $r = 1,\ \dots, N$, and $\vert \cdot \vert$ the absolute value. Note that in \citet{M91} the absolute value was absent and elementary effects could potentially cancel each other out \citep{CCS07}. Therefore, \citet{CCS07} proposed the version presented above, thus making the screening method more robust. A total of $2 \cdot k \cdot N$ model evaluations is needed to compute the full set of sm using the Morris method.

$\mu_i^\ast$ and $\sigma_i$ can now be used to identify non-influential model inputs. Uninfluential inputs exhibit a $\mu_i^\ast$ close to zero. If $\mu_i^\ast$ is large, it depends on $\sigma_i$ whether there exist substantial non-linear or interaction effects. A low $\sigma_i$ indicates that non-linear effects are non-existent, whereas a high $\sigma_i$ suggests large interaction or non-linear effects \citep{GM17}.

The Morris method exhibits some drawbacks. Firstly, as they stand, the sensitivity indices derived by the Morris method are not suited for screening inputs under dependence. To see why consider two inputs $X_i$ and $X_j$ which are dependent, i.e. $G(x_i,\ x_j) \neq G(x_i)G(x_j)$, where $G(\cdot)$ again denotes the cumulative distribution function. If $x_i$ changes, $x_j$ should change as well due to the dependence between the two inputs. The sensitivity indices presented above are derived using a One-At-a-Time approach that does not allow for the screening of dependent inputs \citep{GM17}.

Secondly, similar to the Sobol' indices, the person conducting sa has to take two indices per input into account. Compare to the arguments made in \cref{var_based_sa}.

Thirdly, there exists no clear interpretation of the absolute values of the sensitivity indices. They only provide a ranking of inputs and give a hint of which inputs are the least influential ones \citep{GM17}.

On the advantages, Morris indices are easily computed, with a much lower computational burden than the Shapley effects as presented in \cref{comp_shap}. Recall that Shapley effects as computed by use of the algorithm in \citet{SNS16} came at a cost of $N_V+m \cdot N_I \cdot N_O \cdot (k-1)$ model evaluations. Even the more efficient approach by \citet{PRB20} needed $2^k$ model runs.

[Can learn sth. about model: whether there are interaction and/or non lin. effects. Why? See Morris (1991)? ]

[Especially useful if assump.s about Latin hypercube not justifiable Morris 1991 p. 173]

\citet{BP16} group the Morris method to the family of local sm. However, while the elementary effects themselves consider only local changes, the actual measures for input importance, $\mu_i^\ast$ and $\sigma_i$, average over these $N$ elementary effects. Thus, they take $N$ local changes per input $i$ into account \citep{M91}. Indeed, \citet{CCS11} make a case for the Morris method to be seen a global sm. \citet{BP16} acknowledge that screening methods like the Morris sm stand apart from other local sm.

\subsection{Algorithm for Extended Morris Method}

Analogously to the first-order and total Sobol' indices under dependence, \citet{GM17} developed the following elementary effects for dependent inputs.
\begin{align}
EE_i^{ind} = \frac{f(\bar{x_i}',\ x_{-i}) - f(x_i,\ x_{-i})}{\Delta},\\
EE_i^{full} = \frac{f(x_i',\ \bar{x_{-i}}) - f(x_i,\ x_{-i})}{\Delta},
\end{align}

\noindent where
\begin{itemize}
\item $EE_i^{ind}$ denotes \textit{independent} elementary effects for input $i$, effects that exclude the contributions attributable to the dependence between input $X_i$ and $X_j$ for $i \neq j$, and
\item $EE_i^{full}$ denotes \textit{full} elementary effects for input $i$, that include the effects due to correlation with other inputs.
\end{itemize}

\subsubsection{Sampling Design and Transformation}

\textbf{Radial Design}

\textbf{Inverse Nataf Transformation}

\subsubsection{Computation}

\subsection{Morris Indices for the Rust Model}